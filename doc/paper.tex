\documentclass[acmsmall,natbib=false]{template/acmart}

\AtBeginDocument{\providecommand\BibTeX{{Bib\TeX}}}
\setcopyright{acmlicensed}
\copyrightyear{2025}
\acmYear{2025}
\acmDOI{XXXXXXX.XXXXXXX}

\acmJournal{JACM}
\acmVolume{37}
\acmNumber{4}
\acmArticle{111}
\acmMonth{8}

\RequirePackage[datamodel=acmdatamodel,style=acmauthoryear]{biblatex}

\addbibresource{references}

\begin{document}

\title{I/O Patterns and Bottlenecks in Deep Learning Workloads}

\author{Pablo Alessandro Santos Hugen}
\email{pablo.hugen@inf.ufrgs.br}
\orcid{0009-0005-9046-075X}
\affiliation{
    \institution{Institute of Informatics -- UFRGS}
	\city{Porto Alegre}
	\state{Rio Grande do Sul}
	\country{Brazil}
}

\renewcommand{\shortauthors}{Trovato et al.}

\begin{abstract}
	TODO
\end{abstract}


\keywords{}

\maketitle

\section{Introduction}\label{sec:intro}

% Contextualization

% Problematic

% Objective

% Related Work

\section{Background}\label{sec:background}

% Brief Overview of Machine Learning and Deep Learning Algos

% Brief Overview of I/O into Machine learning algos

\section{Experimental setup}\label{sec:methods}

% Explain DLIO benchmark

% Explain the setup in PCAD

% Explain the Workloads and configurations

\section{Results}\label{sec:results}

% For each benchmark explain the I/O metrics obtained

% Plots and tables

\section{Conclusions}\label{sec:results}

% Summarize the problems and patterns

% Present future works and directions

\printbibliography
\appendix
\end{document}
